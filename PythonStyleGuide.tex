\documentclass[11pt]{article}

\usepackage{amsmath}
\usepackage{amssymb}
\usepackage{fancyhdr}
\usepackage[margin=1in]{geometry}
\usepackage{microtype}
\usepackage{graphicx}
\usepackage{tikz}
\usepackage{listings}
\graphicspath{ {images/} }

\newcommand{\question}[2] {\vspace{.25in} \hrule\vspace{0.5em}
  \noindent{\bf #1: #2} \vspace{0.5em}
  \hrule \vspace{.10in}}
\renewcommand{\part}[1] {\vspace{.10in} {\bf (#1)}}

\newcommand{\myname}{Kyle Vedder}
\newcommand{\myaddress}{\{kvedder, aaronweiss, hpoddar, ilevyor, jholzman,
  jreedie, timothymcnam\}@umass.edu, gordon@cs.umass.edu}
\newcommand{\myhwnum}{4}

\setlength{\parindent}{0pt}
\setlength{\parskip}{5pt plus 1pt}

\pagestyle{fancyplain}
\lhead{\fancyplain{}{\textbf{CS 220 Python Style Guide}}}
\rhead{\fancyplain{}{}}
\chead{\fancyplain{}{}}
\usepackage{color}

\newenvironment{answer}{\color{blue}\ttfamily}{\par}

\begin{document}

\medskip

\thispagestyle{plain}
\begin{center}
  {\Large CS 220 Python Style Guide} \\
  \myaddress\\
  \today \\
\end{center}

\question{0}{Why?}

Style matters. Code is read far more than it's written, and thus writing
readable code is critical.

This document aims to outline guidelines that will make your code more readable;
it's for your TAs and for your instructor, but it's also for you. Debugging is
easier when you can actually read the hodgepodge of code in front of you.

This document ultimately is a very pared down version of the PEP8 style guide.
It's short, please read it.

\question{1}{Indentation}
\begin{enumerate}
  \item \textbf{Use four spaces, not tabs.} Do not mix tabs and spaces; Python 3 does not allow
    for mixing of the two and will throw errors.
  \item \textbf{Use hanging indents for long lines:}

\begin{lstlisting}[language=Python]
# Aligned with opening delimiter.
foo = long_function_name(var_one, var_two,
                         var_three, var_four)
\end{lstlisting}
                       
Note that the variables line up under one another so that it's clear they are
part of the function invocation.

  \item \textbf{Line limit of 80 characters.} Don't write crazy long lines.
\end{enumerate}

\question{2}{Whitespace}

\begin{enumerate}
  \item \textbf{Avoid extraneous whitespace.} Don't put spaces between the start
    of a function parameter list and the first parameter and other places where
    it don't aid readability.
  \item \textbf{Use whitespace to aid in readability.} Add it after a comma in a
    list, and places of that nature that aid in clarifying the code's intent.
\end{enumerate}

\question{3}{Comments}

\begin{enumerate}
  \item \textbf{Use comments sparingly.} Don't make every other line comments,
    as this results in code that doesn't flow well. Comments should carry their
    weight, not just restate the line below them.
  \item \textbf{Avoid inline comments.} Add comments to the line above, not at
    the end of a line after the code.
  \end{enumerate}

\question{4}{Naming Convention}

\begin{enumerate}
  \item \textbf{Use snake case for variable names.} It's
    \textit{my\_variable\_name}, not \textit{myVariableName}.
  \item \textbf{Use Pascal case for class names.} It's
    \textit{MyClassName}, not \textit{my\_class\_name}.
  \item \textbf{Use all caps for constants.} 'Nuff said.
  \item \textbf{Differentiate between internal and external variables.} Begin
    local variables with an underscore: \textit{\_my\_internal\_name} not
    \textit{my\_internal\_name}.
  \item \textbf{Avoid single letter variables.} The variable \textit{O} doesn't
    tell you very much, does it?
\end{enumerate}  
\end{document}
